\documentclass[a4paper,twoside,11pt]{article}

\usepackage[utf8]{inputenc}
\usepackage[brazil]{babel}
\usepackage[explicit]{titlesec}
\usepackage{enumitem}
\usepackage{amsmath,amsthm,amsfonts,amssymb}
\usepackage{calrsfs}
\usepackage{marvosym}
\usepackage{graphicx}

\makeatletter
\newcommand{\oast}{\mathbin{\mathpalette\make@circled\ast}}
\newcommand{\make@circled}[2]{%
  \ooalign{$\m@th#1\smallbigcirc{#1}$\cr\hidewidth$\m@th#1#2$\hidewidth\cr}%
}
\newcommand{\smallbigcirc}[1]{%
  \vcenter{\hbox{\scalebox{0.77778}{$\m@th#1\bigcirc$}}}%
}
\makeatother

\newtheorem*{enunciado}{Enunciado}
\newtheorem*{lemma}{Lema}
\titleformat{\section}{\normalfont\Large\bfseries}{}{0em}{\clearpage#1\ \thesection}

\begin{document}

\title{MA 148 - Prova 3}
\author{}
\date{}
\maketitle



\section{Questão}
\begin{enunciado}
    Considere o conjunto $C = {0, 1, 2} \subseteq \mathbb{Z}$.
    Defina duas operações binárias em C, denotadas por $\oplus$ e $\odot$, por

    $$m \oast n = r \qquad \text{ se } \qquad m \ast n = 3q + r$$

    com $q \in \mathbb{Z}$, $r \in C$ e $\ast \in \left\{ +, . \right\}$ sendo
    $+$ e $.$ a soma e a multiplicação de $\mathbb{Z}$. Mostre que, com estas
    operações, $C$ se torna um corpo. Pode-se definir ordem em $C$ para
    torná-lo um corpo ordenado?

    \begin{proof}[Resposta]
        $<$TODO$>$
    \end{proof}
\end{enunciado}



\section{Questão}
\begin{enunciado}
    Mostre que $\mathbb{Z}$ é ilimitado tanto inferior quanto superiormente em
    $\mathbb{R}$.

    \begin{proof}[Resposta]
        $<$TODO: Corrigir a partir do que o Enos disse$>$
    \end{proof}
\end{enunciado}



\section{Questão}
\begin{enunciado}
    Considere o conjunto $A = \left\{ x \in \mathbb{Q} : x > 0 \text{ e } x^2 > 2 \right\}$.
    Mostre que A não tem mínimo.

    \begin{proof}[Resposta]
        Veja que A tem mínimo $\iff \exists m \in A \text{ tal que } m \leq x \enspace \forall x \in A$.

        \begin{lemma}
            Para todo $a \in A$, existe um $b \in A$ tal que $b < a$.

            \begin{proof}
                Veja que $a^2 > 2 \implies a^2 = 2 + k$, $k \in \mathbb{Q}_{>0}$.
                Assim, seja $b = a - \frac{k}{2a}$. Vejamos que $b \in \mathbb{Q}_{>0}$.
                
                Note que $b > 0 \iff a > \frac{k}{2a}$. Mas veja que com certeza vale que
                $a > \frac{k}{2a}$, pois supondo $a \leq \frac{k}{2a}$:
                \begin{equation} \label{eq:b>0}
                    \begin{split}
                         a \leq \frac{k}{2a} \iff 2a^2 \leq k \implies a^2 \leq k \text
                         \qquad 
                    \end{split}
                \end{equation}

                Mas, por hipótese, $a^2 = k + 2$, ou seja, $a^2 > k$, e \ref{eq:b>0} gera
                contradição.

                Como $b$ é dado puramente por soma e divisão de números racionais, vale também que
                $b \in \mathbb{Q}$. \qed

                Veja também que:

                \begin{equation} \label{eq:b^2_>_2}
                    \begin{split}
                        b^2 &= (a - \frac{k}{2a})^2 \\
                            &= a^2 - k + \frac{k^2}{4a} \\
                            &> a^2 - k = 2
                    \end{split}
                \end{equation}

                Por \ref{eq:b>0} e \ref{eq:b^2_>_2}, temos que $b > 0$ e $b^2 > 2$, logo,
                $b \in A$. Mas, visto que $k > 0$ e $a > 0 \implies \frac{k}{2a} > 0$,
                também vale que $b < a$. \qedhere
            \end{proof}
        \end{lemma}

        Mas veja que, pelo lema, não pode existir
        $m \in A \text{ tal que } m \leq x \enspace \forall x \in A $, pois
        $m \in A \implies \exists m^{\prime} \in A < m$. Logo, $A$ não tem mínimo.
    \end{proof}
\end{enunciado}



\section{Questão}
\begin{enunciado}
    Suponha que $(a_n)_{n \in \mathbb{N}}$ seja uma sequência de números racionais
    e considere a sequência $b_n = a_{n + 148}$, $n \geq 0$. Mostre que $(a_n)_{n \in \mathbb{N}}$
    converge se, e somente se, $(b_n)_{n \in \mathbb{N}}$ convergir.

    \begin{proof}[Resposta]
        $<$TODO$>$
    \end{proof}
\end{enunciado}



\section{Questão}
\begin{enunciado}
    Considere a sequência $a_n = \frac{n - 1}{n}$, $n \geq 1$. Encontre $n_0 \in \mathbb{N}$
    tal que $0,999 < a_n < 1,001$ para todo $n \geq n_0$.

    \begin{proof}[Resposta]
        $<$TODO$>$
    \end{proof}
\end{enunciado}



\section{Questão}
\begin{enunciado}
    Mostre que a sequência $a_n = \sqrt{\frac{9n^2 + 3n - 2}{4n^2+4n}}$, $n \geq 1$, converge e
    encontre seu limite.
    
    \begin{proof}[Resposta]
        $<$TODO$>$
    \end{proof}
\end{enunciado}



\section{Questão}
\begin{enunciado}
    Considere as seguintes sequências $(a_n)_{n \in \mathbb{N}^*}$ e $(b_n)_{n \in \mathbb{N}^*}$:

    $$a_n = \dfrac{1}{2^{k + 1}}\quad\text{sendo } k \in \mathbb{Z} \text{ tal que}\quad 2^k < n \leq 2^{k+1} \qquad \text{e} \qquad b_n = \sum_{k=1}^n \dfrac{1}{k}$$

    \begin{enumerate}[label=\alph*)]
        \item Mostre que $(a_n)_{n \in \mathbb{N}}$ é não crescente e 
            $(b_n)_{n \in \mathbb{N}}$ é crescrente.

        \item Verifique que $a_n \leq \frac{1}{n}$ para todo $n \geq 1$ e use este fato
        para mostrar que $$\lim_{n \rightarrow \infty}{b_n} = +\infty.$$
    \end{enumerate}

    \begin{proof}[Resposta]
        \begin{enumerate}[label=\alph*)]
            \item 
            % % $(a_n)_{n \in \mathbb{N}}$ é não crescente $\iff a_{n+1} \leq a_n$. Veja que
            % $$
            %     n \leq 2^{k+1} \iff a_n \leq \frac{1}{n}
            % $$
            % $$\text{e}$$
            % $$
            %     2^k < n \iff 2^{k+1} < 2n \iff a_n > \frac{1}{2n}
            % $$
            % $$\text{Logo,}\qquad \frac{1}{2n} < a_n \leq \frac{1}{n} \text{, \quad ou ainda } \quad n < 2n^2a_n \leq 2n$$

            % % Dessa forma, $\frac{1}{2n} < a_n \leq \frac{1}{n}$. Veja que se $n = 1$ ou $n = 2$ \footnotemark[1], segue que
            % % $a_{n+1} \leq a_n$:

            % % $$\frac{1}{4} < a_2 \leq \frac{1}{2} < a_1 \leq 1 \implies a_2 < a_1$$
            % % $$\frac{1}{6} < a_3 \leq \frac{1}{3} \quad \text{e} \quad \frac{1}{4} < a_2 \leq \frac{1}{2} \implies a_2 < a_1$$
            % Suponhamos $a_{n + 1} > a_n$. Para $a_{n + 1}$, temos:

            % \begin{align*}
            %     &(2n^2 + 4n + 2) . a_{n+1} \leq 2n + 2\\
            %     &\iff 2n^2.a_{n+1} + 4n.a_{n+1} + 2.a_{n+1} \leq 2n + 2\\
            %     &\iff n^2.a_{n+1} + 2n.a_{n+1} + a_{n+1} \leq n + 1\\
            %     &\iff n^2.a_{n+1} \leq n + 1 - 2n.a_{n+1} - a_{n+1}
            % \end{align*}

            % $$n^2a_n > \frac{n}{2}$$
            $(a_n)_{n \in \mathbb{N}}$ é crescente se e somente se:

            \begin{equation} \tag{1} \label{hypot}
                \begin{split}
                a_{n+1} > a_n \iff \dfrac{1}{2^{a+1}} > \dfrac{1}{2^{b+1}},2^a < n + 1 \leq 2^{a + 1}\quad\text{e}\quad2^b < n \leq 2^{b + 1}\\
                a, b \in \mathbb{Z}
                \end{split}
            \end{equation}

            Mas \begin{equation} \tag{2} \label{corollary}
                \dfrac{1}{2^{a+1}} > \dfrac{1}{2^{b+1}} \iff 2^{a+1} < 2^{b+1} \iff 2^a < 2^b
            \end{equation}
            
            Por (\ref{hypot}), $2^b < n$ e $n + 1 \leq 2^{a + 1}$, e portanto $2^b < n + 1 \leq 2^{a+1}$.
            Unindo isso a (\ref{corollary}), segue que:

            \begin{align*}
                2^a < 2^b &< n + 1 \leq 2^{a + 1} < 2^{b+1} \implies 2^a < 2^b < 2^{a + 1}\\
                &\implies 1 < \dfrac{2^b}{2^a} < 2 \iff 1 < 2^{b-a} < 2
            \end{align*}

            Mas se $a, b \in \mathbb{Z}$, segue que:
            \begin{enumerate}
                \item Se $a = b$, então $2^{b-a} = 1$;
                \item Se $a > b$, então $b - a \leq -1 \implies b - a = -1 - k, k \in \mathbb{Z}_{>0}$, e $2^{-1-k} = \frac{1}{2^{k+1}} < 1$;
                \item Se $a < b$, então $b - a \geq 1 \implies b - a = 1 + k, k \in \mathbb{Z}_{>0}$, e $2^{1+k} = 2 + 2^k > 2$.
            \end{enumerate}
            
            De todo modo, $1 \geq 2^{b-a}$ ou $2^{b-a} > 2$. \Lightning $a$ não é crescente $\implies a$ é não crescente. \qed

            $(b_n)_{n \in \mathbb{N}}$ é crescente se e somente se:
            
            \begin{align*}
                b_{n+1} > b_n &\iff \sum_{k=1}^{n+1} \dfrac{1}{k} > \sum_{k=1}^n \dfrac{1}{k}\\
                &\iff \sum_{k=1}^n \dfrac{1}{k} + \dfrac{1}{n+1} > \sum_{k=1}^n \dfrac{1}{k}
            \end{align*}

            De fato, como $n \in \mathbb{N}^*$, evidentemente $n + 1 \in \mathbb{N}^*$ e, assim, $\dfrac{1}{n+1} > 0$.
            Portanto, segue que $b_{n+1} > b_n$ e assim que $b$ é crescente. \qed

            \item 
        \end{enumerate}
    \end{proof}
\end{enunciado}

\footnote[1]{
    Não existe $a_0$, pois $\nexists k \in \mathbb{Z}$ tal que $2^k < 0$. 
    Até por isso acho que no exercício ele pede para $\mathbb{N}^*$.
}

\end{document}