\documentclass[a4paper,twoside,11pt]{article}

\usepackage[utf8]{inputenc}
\usepackage[brazil]{babel}
\usepackage[explicit]{titlesec}
\usepackage{enumitem}
\usepackage{amsmath,amsthm,amsfonts,amssymb}
\usepackage{calrsfs}
\usepackage{marvosym}
\usepackage{graphicx}

\makeatletter
\newcommand{\oast}{\mathbin{\mathpalette\make@circled\ast}}
\newcommand{\make@circled}[2]{%
  \ooalign{$\m@th#1\smallbigcirc{#1}$\cr\hidewidth$\m@th#1#2$\hidewidth\cr}%
}
\newcommand{\smallbigcirc}[1]{%
  \vcenter{\hbox{\scalebox{0.77778}{$\m@th#1\bigcirc$}}}%
}
\makeatother

\newtheorem*{enunciado}{Enunciado}
\newtheorem*{lemma}{Lema}
\titleformat{\section}{\normalfont\Large\bfseries}{}{0em}{\clearpage#1\ \thesection}

\begin{document}

\title{MA 148 - Prova 3}
\author{}
\date{}
\maketitle



\section{Questão}
\begin{enunciado}
    Considere o conjunto $C = {0, 1, 2} \subseteq \mathbb{Z}$.
    Defina duas operações binárias em C, denotadas por $\oplus$ e $\odot$, por

    $$m \oast n = r \qquad \text{ se } \qquad m \ast n = 3q + r$$

    com $q \in \mathbb{Z}$, $r \in C$ e $\ast \in \left\{ +, . \right\}$ sendo
    $+$ e $.$ a soma e a multiplicação de $\mathbb{Z}$. Mostre que, com estas
    operações, $C$ se torna um corpo. Pode-se definir ordem em $C$ para
    torná-lo um corpo ordenado?

    \begin{proof}[Resposta]
        $<$TODO$>$
    \end{proof}
\end{enunciado}



\section{Questão}
\begin{enunciado}
    Mostre que $\mathbb{Z}$ é ilimitado tanto inferior quanto superiormente em
    $\mathbb{R}$.

    \begin{proof}[Resposta (Enos)]
        Suponha que $\mathbb{Z}$ seja limitado superiormente em $\mathbb{R}$.
        Como $\mathbb{R}$ é corpo completo, existe $\alpha \in \mathbb{R}$
        tal que $\alpha = \sup \mathbb{Z}$.

        Veja que $\alpha - 1 < \alpha$, e portanto não é cota superior de
        $\mathbb{Z}$, pois $\alpha$ é a menor das cotas superiores.

        Mas se $\alpha - 1$ não é cota superior de $\mathbb{Z}$, então
        $\exists \beta \in \mathbb{Z}$ tal que $\beta > \alpha - 1$, e portanto
        $\beta > \alpha$. \Lightning \quad $\mathbb{Z}$ não é limitado
        superiormente em $\mathbb{R}$.

        Suponha agora que $\mathbb{Z}$ seja limitado inferiormente em
        $\mathbb{R}$. Nesse caso, existe $\gamma \in \mathbb{R}$ tal que
        $\gamma \leq \beta$ $\forall \beta \in \mathbb{Z}$.

        Mas se existisse tal $\gamma$, valeria que $-\gamma$ é cota superior
        de $\mathbb{Z}$, caso contrário $\exists \beta \in \mathbb{Z}$ tal que
        $-\beta < \gamma$. Porém, como já demonstramos, não existe cota superior
        de $\mathbb{Z}$ em $\mathbb{R}$. \Lightning \quad $\mathbb{Z}$ também 
        não é limitado inferiormente em $\mathbb{R}$.
    \end{proof}
\end{enunciado}



\section{Questão}
\begin{enunciado}
    Considere o conjunto $A = \left\{ x \in \mathbb{Q} : x > 0 \text{ e } x^2 > 2 \right\}$.
    Mostre que A não tem mínimo.

    \begin{proof}[Resposta]
        Veja que A tem mínimo $\iff \exists m \in A \text{ tal que } m \leq x \enspace \forall x \in A$.

        Mas $a^2 > 2 \implies a^2 = 2 + k$, $k \in \mathbb{Q}_{>0}$.
        Assim, seja $b = a - \frac{k}{2a}$. Vejamos que $b \in \mathbb{Q}_{>0}$.
        
        Note que $b > 0 \iff a > \frac{k}{2a}$. Mas veja que com certeza vale que
        $a > \frac{k}{2a}$, pois supondo $a \leq \frac{k}{2a}$:
        \begin{equation} \label{eq:b>0}
            \begin{split}
                    a \leq \frac{k}{2a} \iff 2a^2 \leq k \implies a^2 \leq k \text
                    \qquad 
            \end{split}
        \end{equation}

        Mas, por hipótese, $a^2 = k + 2$, ou seja, $a^2 > k$, e \ref{eq:b>0} gera
        contradição.

        Como $b$ é dado puramente por soma e divisão de números racionais, vale também que
        $b \in \mathbb{Q}$.

        Veja também que:

        \begin{equation} \label{eq:b^2_>_2}
            \begin{split}
                &b^2 = (a - \frac{k}{2a})^2 = a^2 - k + \frac{k^2}{4a} \\
                &\text{Mas } k^2 > 0 \text{ e } 4a > 0\text{, logo } \frac{k^2}{4a} > 0 \text{ e portanto } b^2 > a^2 - k = 2
            \end{split}
        \end{equation}

        Por \ref{eq:b>0} e \ref{eq:b^2_>_2}, temos que $b > 0$ e $b^2 > 2$, logo,
        $b \in A$. Mas, visto que $k > 0$ e $a > 0 \implies \frac{k}{2a} > 0$,
        também vale que $b = a - \frac{k}{2a} < a$.

        Assim, não pode existir
        $m \in A \text{ tal que } m \leq x \enspace \forall x \in A $, pois
        $m \in A \implies \exists m^{\prime} \in A < m$. Logo, $A$ não tem mínimo.
    \end{proof}
\end{enunciado}



\section{Questão}
\begin{enunciado}
    Suponha que $(a_n)_{n \in \mathbb{N}}$ seja uma sequência de números racionais
    e considere a sequência $b_n = a_{n + 148}$, $n \geq 0$. Mostre que $(a_n)_{n \in \mathbb{N}}$
    converge se, e somente se, $(b_n)_{n \in \mathbb{N}}$ convergir.

    \begin{proof}[Resposta]
        $<$TODO$>$
    \end{proof}
\end{enunciado}



\section{Questão}
\begin{enunciado}
    Considere a sequência $a_n = \frac{n - 1}{n}$, $n \geq 1$. Encontre $n_0 \in \mathbb{N}$
    tal que $0,999 < a_n < 1,001$ para todo $n \geq n_0$.

    \begin{proof}[Resposta (Enos)]
        $$a_n = \frac{n - 1}{n}, n \geq 1$$
        $$0.999 < a_n < 1.001 \quad \forall n \geq n_0$$
        \begin{align*}
            &0.999 < a_n < 1.001 \iff 0.999 < \frac{n - 1}{n} < 1.001\\
            &\iff 0.999 < 1 - \frac{1}{n} < 1.001 \iff -0.001 < -\frac{1}{n} < 0.001\\
            &\iff |-\frac{1}{n}| < 0.001 \iff \frac{1}{n} < \frac{1}{1000} \iff n > 1000
        \end{align*}

        Assim, $n_0 = 1001$.
    \end{proof}
\end{enunciado}



\section{Questão}
\begin{enunciado}
    Mostre que a sequência $a_n = \sqrt{\frac{9n^2 + 3n - 2}{4n^2+4n}}$, $n \geq 1$, converge e
    encontre seu limite.
    
    \begin{proof}[Resposta (Igor)]
        Primeiramente, iremos escrever o termo geral de maneira diferente e equivalente.

        $$a_n = \dfrac{\sqrt{9n^2+3n-2}}{\sqrt{4n^2+4n}}.\frac{1/\sqrt{n^2}}{1/\sqrt{n^2}}
        = \dfrac{\sqrt{9+\frac{3}{n}-\frac{2}{n^2}}}{\sqrt{4+\frac{4}{n}}}$$

        Mas
        $\lim_{n \rightarrow \infty}\frac{3}{n}
        = \lim_{n \rightarrow \infty}-\frac{2}{n^2}
        = \lim_{n \rightarrow \infty}\frac{4}{n} = 0$

        Dessa forma,
        
        $$\lim_{n \rightarrow \infty}
        \dfrac{\sqrt{9+\frac{3}{n}-\frac{2}{n^2}}}{\sqrt{4+\frac{4}{n}}}
        = \lim_{n \rightarrow \infty} \dfrac{\sqrt{9+0-0}}{\sqrt{4+0}} = \dfrac{\sqrt{9}}{\sqrt{4}} = \dfrac{3}{2}.$$
        
        Logo, a sequência converge e seu limite é $\dfrac{3}{2}$.
    \end{proof}
\end{enunciado}



\section{Questão}
\begin{enunciado}
    Considere as seguintes sequências $(a_n)_{n \in \mathbb{N}^*}$ e $(b_n)_{n \in \mathbb{N}^*}$:

    $$a_n = \dfrac{1}{2^{k + 1}}\quad\text{sendo } k \in \mathbb{Z} \text{ tal que}\quad 2^k < n \leq 2^{k+1} \qquad \text{e} \qquad b_n = \sum_{k=1}^n \dfrac{1}{k}$$

    \begin{enumerate}[label=\alph*)]
        \item Mostre que $(a_n)_{n \in \mathbb{N}}$ é não crescente e 
            $(b_n)_{n \in \mathbb{N}}$ é crescrente.

        \item Verifique que $a_n \leq \frac{1}{n}$ para todo $n \geq 1$ e use este fato
        para mostrar que $$\lim_{n \rightarrow \infty}{b_n} = +\infty.$$
    \end{enumerate}

    \begin{proof}[Resposta]
        \begin{enumerate}[label=(\alph*)]
            \item $(a_n)_{n \in \mathbb{N}}$ é crescente se e somente se:

            \begin{equation} \tag{1} \label{hypot}
                \begin{split}
                a_{n+1} > a_n \iff \dfrac{1}{2^{a+1}} > \dfrac{1}{2^{b+1}},2^a < n + 1 \leq 2^{a + 1}\quad\text{e}\quad2^b < n \leq 2^{b + 1}\\
                a, b \in \mathbb{Z}
                \end{split}
            \end{equation}

            Mas \begin{equation} \tag{2} \label{corollary}
                \dfrac{1}{2^{a+1}} > \dfrac{1}{2^{b+1}} \iff 2^{a+1} < 2^{b+1} \iff 2^a < 2^b
            \end{equation}
            
            Por (\ref{hypot}), $2^b < n$ e $n + 1 \leq 2^{a + 1}$, e portanto $2^b < n + 1 \leq 2^{a+1}$.
            Unindo isso a (\ref{corollary}), segue que:

            \begin{align*}
                2^a < 2^b &< n + 1 \leq 2^{a + 1} < 2^{b+1} \implies 2^a < 2^b < 2^{a + 1}\\
                &\implies 1 < \dfrac{2^b}{2^a} < 2 \iff 1 < 2^{b-a} < 2
            \end{align*}

            Mas se $a, b \in \mathbb{Z}$, segue que:
            \begin{enumerate}[label=\arabic*.]
                \item Se $a = b$, então $2^{b-a} = 1$;
                \item Se $a > b$, então $b - a \leq -1 \implies b - a = -1 - k, k \in \mathbb{Z}_{>0}$, e $2^{-1-k} = \frac{1}{2^{k+1}} < 1$;
                \item Se $a < b$, então $b - a \geq 1 \implies b - a = 1 + k, k \in \mathbb{Z}_{>0}$, e $2^{1+k} = 2.2^k > 2$.
            \end{enumerate}
            
            De todo modo, $1 \geq 2^{b-a}$ ou $2^{b-a} > 2$. \Lightning $a$ não é crescente $\implies a$ é não crescente. \qed

            $(b_n)_{n \in \mathbb{N}}$ é crescente se e somente se:
            
            \begin{align*}
                b_{n+1} > b_n &\iff \sum_{k=1}^{n+1} \dfrac{1}{k} > \sum_{k=1}^n \dfrac{1}{k}\\
                &\iff \sum_{k=1}^n \dfrac{1}{k} + \dfrac{1}{n+1} > \sum_{k=1}^n \dfrac{1}{k}
            \end{align*}

            De fato, como $n \in \mathbb{N}^*$, evidentemente $n + 1 \in \mathbb{N}^*$ e, assim, $\dfrac{1}{n+1} > 0$.
            Portanto, segue que $b_{n+1} > b_n$ e assim que $b$ é crescente. \qed

            \item Pela definição de $(a_n)_{n \in \mathbb{N}^*}$:
            \begin{equation}\tag{1}\label{an_leq_1/n}
                n \leq 2^{k + 1} \implies \dfrac{1}{2^{k+1}} \leq \dfrac{1}{n}
            \end{equation}
            Logo, $a_n \leq \dfrac{1}{n}$ $\forall n \in \mathbb{N}^*$.
            Seja então $s_a$ a série associada a $a$ dada por:

            \begin{equation}\tag{2}\label{series}
                s_n = \sum^n_{i=1}\dfrac{1}{2^{k+1}} \quad \text{onde} \quad 2^k < i \leq 2^{k+1}, k \in \mathbb{Z}
            \end{equation}

            Por (\ref{an_leq_1/n}), segue que $s_n \leq b_n$ $\forall n \in \mathbb{N}^*$.

            % Veja que $2s_n = \sum^n_{i=1}\dfrac{1}{2^k}$. Mas $2^k < i$, logo, certamente
            % $2s_n < b_n$.
            
            % Equivalentemente, $2s_n = \sum^n_{i=1}\dfrac{1}{i - r}$, onde
            % $r = i - 2^k$. Vale que $r > 0$, pois $i > 2^k$.

            Veja que para cada intervalo $(2^k, 2^{k+1}]$, existem
            $2^{k+1} - 2^k = 2^k$ valores distintos de $i$ para os quais $i$ está
            contido no intervalo. Também vale que todos tais intervalos são
            disjuntos de todo outro intervalo com a mesma forma.

            Assim, é justo afirmar que $s_n$ para $n \in \left\{2^{\alpha+1} : \alpha \in \mathbb{N}\right\}$
            vale

            \begin{equation}\label{crazyness}
                \begin{split}
                    s_n &= s_{2^a} + 2^\alpha . \dfrac{1}{2^{\alpha+1}}
                        = s_{2^a} + \frac{1}{2}
                \end{split}
            \end{equation}
            
            No caso em que $n = 1$, por (\ref{series}),
            $s_n = \sum^1_{i=1}\dfrac{1}{2} = \dfrac{1}{2}$,
            pois a única escolha de $k \in \mathbb{Z}$ tal que
            $2^k < 1 \leq 2^{k+1}$ é justamente $1$ (Base de indução).

            \paragraph{H.I.} $s_n > \alpha$ para algum $n \in \left\{2^{\alpha+1} : \alpha \in \mathbb{N}\right\}$
            \paragraph{T.I.} $s_{4n} > \alpha + 1$
            \footnote{
                Minha primeira ideia foi usar 2n,
                mas estava dando $>\alpha + \frac{1}{2}$.
                Dobrando fica $> \alpha + 1$ e ajeita a indução
            }

            Veja que $2^a = \frac{n}{2}$. Com isso e (\ref{crazyness}), segue
            que $s_{2n} = s_n + \frac{1}{2} > \alpha + \frac{1}{2}$, e por
            conseguinte $s_{4n} = s_{2n} + \frac{1}{2} > \alpha + 1$. \qed

            Assim, $\exists n \in \left\{2^{\alpha+1} : \alpha \in \mathbb{N}\right\}$
            tal que $s_n > \alpha$ $\forall \alpha \in \mathbb{N}$.

            Mas, dessa forma, claramente $\exists n \in \left\{2^{\alpha+1} : \alpha \in \mathbb{N}\right\}$
            tal que $b_n > \alpha$ $\forall \alpha \in \mathbb{N}$, pois $b_n \geq s_n$.

            Portanto, como $\mathbb{R}$ é arquimediano, 
            $\exists \alpha \in \mathbb{N}$ tal que
            $\alpha > r$ $\forall r \in \mathbb{R}$, segue que
            $b_n > r$ $\forall r \in \mathbb{R}$. Assim,
            $\lim_{n \rightarrow \infty} b_n = +\infty$.
        \end{enumerate}
    \end{proof}
\end{enunciado}

\end{document}