\documentclass[a4paper,twoside,11pt]{article}

\usepackage[utf8]{inputenc}
\usepackage[brazil]{babel}
\usepackage[explicit]{titlesec}
\usepackage{enumitem}
\usepackage{amsmath,amsthm,amsfonts,amssymb}
\usepackage{calrsfs}
\usepackage{marvosym}

\newtheorem*{enunciado}{Enunciado}
\newtheorem*{lemma}{Lema}
\titleformat{\section}{\normalfont\Large\bfseries}{}{0em}{\clearpage#1\ \thesection}

\begin{document}

\title{MA 148 - Prova 3}
\author{}
\date{}
\maketitle



\section{Questão}
\begin{enunciado}
    Considere o conjunto $C = {0, 1, 2} \subseteq \mathbb{Z}$.
    Defina duas operações binárias em C, denotadas por $\oplus$ e $\odot$, por

    $$m \ast n = r \qquad \text{ se } \qquad m \ast n = 3q + r$$

    com $q \in \mathbb{Z}$, $r \in C$ e $\ast \in \left\{ +, . \right\}$ sendo
    $+$ e $.$ a soma e a multiplicação de $\mathbb{Z}$. Mostre que, com estas
    operações, $C$ se torna um corpo. Pode-se definir ordem em $C$ para
    torná-lo um corpo ordenado?

    \begin{proof}[Resposta]
        $<$TODO$>$
    \end{proof}
\end{enunciado}



\section{Questão}
\begin{enunciado}
    Mostre que $\mathbb{Z}$ é ilimitado tanto inferior quanto superiormente em
    $\mathbb{R}$.

    \begin{proof}[Resposta]
        Suponhamos que $\mathbb{Z}$ seja limitado inferior ou superiormente em $\mathbb{R}$.
        Isso vale $\iff \exists r_1, r_2 \in \mathbb{R}$ tal que $r_1 < q$ e $r_2 > q$
        $\forall q \in \mathbb{Q}$, pela definição de conjunto limitado superior e
        inferiormente.

        Vejamos que isso não pode ser válido.

        \begin{lemma}
            $\forall r \in \mathbb{R}_{\geq 0} \enspace (\exists n \in \mathbb{N} \text{ tal que } n > r$)

            \begin{proof}
                Veja que, para todo $r < 1$, o Lema vale, pois $1 \in \mathbb{N}$ e $1 > r$. Mas $r > 1 \iff r = a + k$,
                $0 < a < 1$, $k > 1 - a$.
                
                Seja $A = \left\{ x \in \mathbb{R} : x > 1 - a \right\}$ e $k = \inf A$. Observe que $1 - a > 0 \implies k > 0$,
                e $1 - \frac{a}{2}$ é menor que $1$ e maior que $1 - a$, ou seja, $1$ está contido em $A$ mas não é
                ínfimo de $A \implies k = \inf A < 1$.

                Mas se $k \in \mathbb{R}_{\geq 0}$ e $k < 1$, sabemos que existe um $k^{\prime} \in \mathbb{N}$ tal que
                $k^{\prime} > k$. Dessa forma, para todo $r > 1$, $k^{\prime} + a > r \implies k^{\prime} + 1 > r$.
                
                Como $k^{\prime} \in \mathbb{N}$, vale que $k^{\prime} + 1 \in \mathbb{N}$ também. Logo,
                para todo $r \in \mathbb{R}_{\geq 0}$ existe um $n \in \mathbb{N}$ tal que $n > r$.
            \end{proof}
        \end{lemma}

        Pelo lema, temos que $\mathbb{N} = \mathbb{Z}_{\geq 0}$ é ilimitado superiormente em $\mathbb{R}_{\geq 0}$.
        Mas veja que isso também implica que $\mathbb{Z}$ é ilimitado inferiormente em $\mathbb{R}_{\leq 0}$, pois para
        todo par $(r, z) \in \mathbb{R}_{\leq 0} \times \mathbb{Z}_{\leq 0}$, $r < z \iff -r > -z$. Como
        $(-r, -z) \in \mathbb{R}_{\geq 0} \times \mathbb{Z}_{\geq 0}$, sabemos que para todo $r \in \mathbb{R}_{\leq 0}$
        existe par $(-r, -z)$ tal que $-r < -z$ e, com isso, que $\mathbb{Z}_{\leq 0}$ não é limitado inferiormente
        em $\mathbb{R}_{\leq 0}$.

        Claramente, $\mathbb{Z}_{\geq 0}$ não é limitado superiormente em $\mathbb{R}_{\leq 0}$ e nem $\mathbb{Z}_{\leq 0}$
        é limitado inferiormente em $\mathbb{R}_{\geq 0}$.

        De todo modo, para todo $r \in \mathbb{R}$, existe $z \in \mathbb{Z}$ tal que $z > r$ ou $z < r$, e sendo assim
        $\mathbb{Z}$ não é limitado em $\mathbb{R}$.

    \end{proof}
\end{enunciado}



\section{Questão}
\begin{enunciado}
    Considere o conjunto $A = \left\{ x \in \mathbb{Q} : x > 0 \text{ e } x^2 > 2 \right\}$.
    Mostre que A não tem mínimo.

    \begin{proof}[Resposta]
        Veja que A tem mínimo $\iff \exists m \in A \text{ tal que } m \leq x \enspace \forall x \in A$.

        \begin{lemma}
            Para todo $a \in A$, existe um $b \in A$ tal que $b < a$.

            \begin{proof}
                Veja que $a^2 > 2 \implies a^2 = 2 + k$, $k \in \mathbb{Q}_{>0}$.
                Assim, seja $b = a - \frac{k}{2a}$. Vejamos que $b \in \mathbb{Q}_{>0}$.
                
                Note que $b > 0 \iff a > \frac{k}{2a}$. Mas veja que
                \begin{equation} \label{eq:b>0}
                    \begin{split}
                         a \leq \frac{k}{2a} \iff 2a^2 \leq k \implies a^2 \leq k \text
                         \qquad \text{\Lightning} \quad a > \frac{k}{2a} \implies b > 0
                    \end{split}
                \end{equation}
                
                Como $b$ é dado puramente por soma e divisão de números racionais, vale também que
                $b \in \mathbb{Q}$. \qed

                Veja também que:

                \begin{equation} \label{eq:b^2_>_2}
                    \begin{split}
                        b^2 &= (a - \frac{k}{2a})^2 \\
                            &= a^2 - k + \frac{k^2}{4a} \\
                            &> a^2 - k = 2
                    \end{split}
                \end{equation}

                Por \ref{eq:b>0} e \ref{eq:b^2_>_2}, temos que $b > 0$ e $b^2 > 2$, logo,
                $b \in A$. Mas, visto que $k > 0$ e $a > 0 \implies \frac{k}{2a} > 0$,
                também vale que $b < a$. \qedhere
            \end{proof}
        \end{lemma}

        Mas veja que, pelo lema, não pode existir
        $m \in A \text{ tal que } m \leq x \enspace \forall x \in A $, pois
        $m \in A \implies \exists m^{\prime} \in A < m$. Logo, $A$ não tem mínimo.
    \end{proof}
\end{enunciado}



\section{Questão}
\begin{enunciado}
    Suponha que $(a_n)_{n \in \mathbb{N}}$ seja uma sequência de números racionais
    e considere a sequência $b_n = a_{n + 148}$, $n \geq 0$. Mostre que $(a_n)_{n \in \mathbb{N}}$
    converge se, e somente se, $(b_n)_{n \in \mathbb{N}}$ convergir.

    \begin{proof}[Resposta]
        $<$TODO$>$
    \end{proof}
\end{enunciado}



\section{Questão}
\begin{enunciado}
    Considere a sequência $a_n = \frac{n - 1}{n}$, $n \geq 1$. Encontre $n_0 \in \mathbb{N}$
    tal que $0,999 < a_n < 1,001$ para todo $n \geq n_0$.

    \begin{proof}[Resposta]
        $<$TODO$>$
    \end{proof}
\end{enunciado}



\section{Questão}
\begin{enunciado}
    Mostre que a sequência $a_n = \sqrt{\frac{9n^2 + 3n - 2}{4n^2+4n}}$, $n \geq 1$, converge e
    encontre seu limite.
    
    \begin{proof}[Resposta]
        $<$TODO$>$
    \end{proof}
\end{enunciado}



\section{Questão}
\begin{enunciado}
    Considere as seguintes sequências $(a_n)_{n \in \mathbb{N}^*}$ e $(b_n)_{n \in \mathbb{N}^*}$:

    $$a_n = \dfrac{1}{2^{k + 1}}\quad\text{sendo } k \in \mathbb{Z} \text{ tal que}\quad 2^k < n \leq 2^{k+1} \qquad \text{e} \qquad b_n = \sum_{k=1}^n \dfrac{1}{k}$$

    \begin{enumerate}[label=\alph*)]
        \item Mostre que $(a_n)_{n \in \mathbb{N}}$ é não crescente e 
            $(b_n)_{n \in \mathbb{N}}$ é crescrente.

        \item Verifique que $a_n \leq \frac{1}{n}$ para todo $n \geq 1$ e use este fato
        para mostrar que $$\lim_{n \rightarrow \infty}{b_n} = +\infty.$$
    \end{enumerate}

    \begin{proof}[Resposta]
        $<$TODO$>$
    \end{proof}
\end{enunciado}

\end{document}